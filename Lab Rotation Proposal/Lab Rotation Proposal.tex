\documentclass[fontsize=11pt, twocolumn]{scrartcl}	 % A4 paper and 11pt font size
\usepackage[svgnames]{xcolor} % Using colors
\usepackage{background} % To include background images
\usepackage{fancyhdr} % Needed to define custom headers/footers
\usepackage[english]{babel} % English language/hyphenation
\usepackage[a4paper, includehead, headheight=0.6cm, inner=2cm ,outer=2cm, top=2.5 cm, bottom=3cm]{geometry}  % Changing size of document
%\usepackage[square, numbers, comma, sort&compress]{natbib} % Use the natbib reference package - read up on this to edit the reference style; if you want text (e.g. Smith et al., 2012) for the in-text references (instead of numbers), remove 'numbers' 


%%%%%% Setting up the stye

\setlength\parindent{0pt} % Gets rid of all indentation
\backgroundsetup{contents={\includegraphics[width=\textwidth]{logo.jpg}},scale=1,placement=top,opacity=0.4,position={8.3cm,2cm}} %  OIST Logo

\pagestyle{fancy} % Enables the custom headers/footers

\lhead{} \rhead{} % Headers - all  empty

\rfoot{ \color{Grey} OIST Graduate School }
\cfoot{}

\title{\vspace{-2.8cm}  \color{DarkRed} Laboratory Rotation Proposal} 
\subtitle{Title of the proposal % Title of the proposal
\vspace{-2cm} }
\date{} % No date

\renewcommand{\headrulewidth}{0.0pt} % No header rule
\renewcommand{\footrulewidth}{0.4pt} % Thin footer rule

%%%%%% Starting the document

\begin{document}

\maketitle % Print the title
\thispagestyle{fancy} % Enabling the custom headers/footers for the first page 

% In the following lines, add the relevant information
\textbf{Student Name:}  

\textbf{Student ID:} 

\textbf{Date of Submission:}

 \textbf{Unit Professor:} 

\textbf{Unit Name:} 	

\subsection*{Instructions for Laboratory Rotation Proposal}

Please submit a one-page rotation proposal, preferably in PDF format.  Use one page immediately following this cover sheet.

This should be written entirely by you, and generated after consultation with your laboratory rotation supervisor.  Keep strictly to the 1-page limit, including any necessary figures and references (that's part of the learning outcome of each rotation: making a succinct proposal for research).

You have three weeks to submit, until the end of the first calendar month of term.  Extensions may be granted if you change rotation lab (with permission, using the form ``Request to Change Lab Rotation" available on the Graduate School website) or if you start term late due to absence from OIST.

\vfill \break

\subsection*{Specific training requirements provided by non-faculty member}

It is expected that the primary supervisor is the faculty member in charge of the research unit. In special circumstances, direct supervision by another member of the unit may be permitted  (e.g. training in a method by a postdoc or technician in the laboratory).  Such supervision must be described and acknowledged in advance where possible.

\begin{itemize}
\item{Name of provider} 
\item{Description of training provided}
\end{itemize}

\subsection*{Training or Health Check needed}

\subsection*{special resources, fieldwork, or business travel}

Describe those on a separate page


\end{document}
