%----------------------------------------------------------------------------------------
%	GLOSSARY
%----------------------------------------------------------------------------------------
% For automatic construction of a list of glossaries, glossaries package is used.
%
% To define a glossary, use the following command.
% 		\newglossaryentry{label}{
%			name=name_of_this_entry,
%			text=text_of_this_entry,
%			plural=plural_of_this_entry,
%			description=description_of_this_entry
%		}
% Here, label is a lebel of an entry.
% name=name_of_this_entry passes a name of this entry to the package.
% text=text_of_this_entry passes a name of this entry in texts to the package. text_of_this_entry is used in texts.
% plural=plural_of_this_entry passes a plural form of this entry in texts to the package. plural_of_this_entry is used in texts.
% description=description_of_this_entry passes description of this entry to the package.
%
% To use this entry in texts, use \gls{label}. If the first letter needs to be capitalized, use \Gls instead.
% If you need a plural form, use either \glspl{label} or \Glspl{label} depending on whether you want to capitalize
% the first letter.
%
% For more details, see the examples below or the following links:
% 	(longest)	http://www.math.ucsd.edu/~jeggers/latex/glossaries-user.pdf
% 	(longer)	https://en.wikibooks.org/wiki/LaTeX/Glossary
% 	(long)		https://ja.overleaf.com/learn/latex/Glossaries

\newglossaryentry{dipole_blockade}{
	name=Dipole Blockade,
	text=dipole blockade,
	plural=dipole blockades,
	description={Phenomenon in which the simultaneous excitation of two atoms is inhibited by their dipolar interaction}
}

\newglossaryentry{cavity_induced_transparency}{
	name=Cavity Induced Transparency,
	text=cavity induced transparency,
	plural=cavity induced transparencies,
	description={Phenomenon in which a cavity containing two atoms excited with light at a frequency halfway between the atomic frequencies contains the number of photons an empty cavity would contain}
}