
\chapter{Guidelines on the preparation of theses} \label{ch-1}

The guidelines below set out the organization and formatting requirements of the OIST PhD thesis, in order to assist students in the preparation of theses for submission.

The academic requirements of the thesis are defined in the Academic Program Policies that you can find here: \url{https://groups.oist.jp/grad/academic-program-policies}. Please always refer to the website for the latest updates in the guidelines as there may be a delay in updating the guidelines in this template.

This particular documents refers specifically a thesis written in \LaTeX. As such, some points from the full guideline (for example page sizes) are not referenced directly here as they are already defined in this template. Some other points concerning specific pages (for example the abstract) are described in the specific pages themselves in this PDF.

\section{Guidelines on the preparation of theses}

\textbf{Plagiarism and Fraud}:  Students are reminded that they must take all necessary precautions to avoid plagiarism and fraudulent misrepresentation of data.  The Graduate School conducts plagiarism checks on all submitted theses, and may require rewriting if present.  When submitting a thesis by dissertation, students should avoid self-plagiarism through rewriting earlier published work and/or self-citation.

\textbf{Reproducibility}: OIST is committed to openness in science.  A cornerstone of this philosophy is reproducibility.  Your thesis should present all data and methods necessary to allow complete repetition of the experiments and their results, and to allow expert review of your analysis of data.  Accordingly, you must ensure that your methods are comprehensive, and that your data sets and code are available for subsequent review by lodging them in the OIST Institutional Repository or some other data repository or database, as appropriate.

\textbf{Inclusion of Published Material}: In some cases, inclusion of published material as chapters is desirable.  Normally, however, when published material is included in the thesis, it should be modified in order to remove redundancy and achieve a coherent narrative.  It is essential to indicate clearly any portion of the thesis that duplicates parts of articles that were previously published by the candidate.  The candidate must cite the article and indicate any parts of a section or chapter of the thesis that depend on the previously published article.  This does not apply to previous documents such as thesis proposals and reports written as part of the candidate’s research.

An appropriate level of independence on the part of the student is expected.  If parts of the thesis are based on published work under joint authorship, the supervisor should provide a statement about the extent to which this is the candidate’s own work as part of the standard supervisor declaration.

When including material from publications in a thesis, students should be aware of the copyright policies of journals.  It is recommended that students request journals to vary their normal copyright agreements to allow material from an article to be included in a thesis (as the thesis will be publicly available through the University’s library).  If, for copyright reasons, material from previously published papers may not be included in the electronically published thesis, the electronically published thesis may cite papers that are already published.

\section{Organization of chapters and sections}

\textbf{Title Page}: This page is the first printed page.

\textbf{Choice of Title}: Select a descriptive and unique title that clearly communicates your research.  Avoid brief or misleading titles.  The title will be displayed on your graduation testamur.  The title should be unique within OIST, to distinguish your thesis from those of others working on similar subject.

\textbf{Declaration of Original Authorship}:  Students must provide a signed declaration that the thesis is their own work and is original.

\textbf{Co-authorship}: Co-authorship is not allowed in an OIST PhD thesis.  All research and analysis is to be the student’s own work.  Where co-authors have contributed to papers arising from the research, this data should not be included unless essential to the scientific narrative.  When included, full disclosure of the contribution is required.  Any and all work conducted by others, either internal or external to OIST, must be acknowledged.

\textbf{Abstract, Acknowledgements, List of Abbreviations, Glossary, Nomenclature, Dedication, Table of Contents, List of Figures and Tables}: Those are commented directly in the template. Glossary, Nomenclature, and Dedication pages are optional.

\textbf{Main body}:  The main body of text may be arranged as a single body of material, divided into subsections of Introduction (including a statement of the problem to be investigated), Methods, Results, Discussions, or, if preferred, in chapters that each deal with a smaller part of the research, each itself divided into subchapters as above.

\textbf{Bibliography}: A complete list of all articles and books cited within the thesis, once only, at the end of the thesis.  Citations should provide the title of the reference, and list at least the first three authors (et al. format is acceptable).  Articles not cited within the thesis should not be included.

\textbf{Appendices}: As required.  Unlike a journal article, no data or discussion may be presented separately as unpublished supplementary documents or data.  Appendices should be used instead for material that is tangentially relevant to the thesis but does not belong in the main narrative.  If reference is needed to large volumes of data that cannot be printed (for example, an annotated genome, or a simulation including moving images), the data should be located on an OIST repository or public database and the URL of the dataset provided in the thesis.


\section{Formatting Requirements}

\textbf{Page size, Margins, Spacing, Justification, Pagination, Header, Fonts}: those are already built-in the template. Do not modify them.

\textbf{Equations}: Equations are considered part of the main text.  As such, they should be formatted consistently throughout the thesis, following the advice of the Thesis Supervisor.  Equations should be numbered to the right-hand margin.

\textbf{Spelling}: American spelling.


\textbf{Colors}: Color may be used in images and charts where necessary to enhance comprehension, but not for normal text or headings.  The combination of red and green for binary images should be avoided to assist those who have difficulty in discerning hues.  All text should be in black unless color-coding is necessary for meaning or contrast.

\textbf{Figues, Tables, Images}: Those are detailed in a later Chapter with examples.

\textbf{Word length}: No minimum word length is imposed on OIST theses.  However, students must be concise in language and succinct in expression.  The average length of a PhD thesis will vary between fields and between authors, but typical PhD theses are 100-400 pages in length (20,000-80,000 words of main body text).

\textbf{Citations}: All papers cited in the thesis must be referenced in a style relevant to the student’s field.  All referencing must include the full title, authors, reference location and the year of publication, all in the same style for all references.  Citation style must be consistent throughout the thesis.  Reference manager software, such as Endnote, or BibTex which offers similar functionality with \LaTeX, may be used. 

Citing one reference can be done like so: \cite{Lee98} and multiple references in one go like so \cite{Fil09, Muc10, Kra27}.

\textbf{Editing}: The thesis must be entirely the work of the student.  Minimal editing may be provided by the Thesis Supervisor(s) or peers, but only as a review of initial drafts.  Assistance should not be sought from OIST internal or paid external editing services unless directed to do so by the Dean of Graduate School in revision stages.

