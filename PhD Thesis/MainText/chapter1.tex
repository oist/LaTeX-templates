
\chapter{Guidelines on the preparation of theses} \label{ch-1}

These guidelines set out the organization and formatting requirements of the OIST PhD thesis, in order to assist students in the preparation of theses for submission. The academic requirements of the thesis are defined in the PRP in section 5.3.13, while the format of the submitted examination and publication versions of the thesis are described here.

This particular documents refers specifically a thesis written in \LaTeX. As such, some points from the full guideline (for example page sizes) are not referenced directly here as they are already defined in this template. Some other points concerning specific pages (for example the abstract) are described in the specific pages themselves in this PDF.

\section{Guidelines on the preparation of theses}

\textbf{Plagiarism and Fraud}:  Students are reminded that they must take all necessary precautions to avoid plagiarism and fraudulent misrepresentations of data. The Graduate School does check each thesis for plagiarism using automated online checks, and we will ask you to rewrite if this is present.  It is your responsibility to ensure that self-plagiarism is avoided as far as possible by rewriting, by self-citation, and by absolutely refraining from copy-paste from earlier articles you or others have published.

\textbf{Reproducibility}: OIST is committed to openness in science, and a cornerstone of this is the concept of reproducibility. Your thesis should present all the data and methods necessary to allow complete repetition of the experiments and their results, and to allow expert review of your analysis of data. Accordingly, you must ensure that your methods are comprehensive, and that your data sets and code are available for subsequent review by lodging them in the OIST Institutional Repository or some other data repository or database, as appropriate.

\textbf{Inclusion of Published Material}: In some cases, inclusion of published material as chapters is desirable. Normally, however, when published material is included in the thesis, it should be modified in order to remove redundancy and achieve a coherent narrative. It is essential to indicate clearly any portion of the thesis that duplicates parts of articles that were previously published by the candidate. The candidate must cite the article and indicate any parts of a section or chapter of the thesis that depend on the previously published article. This does not apply to previous documents such as thesis proposals and reports written as part of the candidate?s research.

An appropriate level of independence on the part of the student is expected. If parts of the thesis are based on published work under joint authorship, the supervisor should provide a statement about the extent to which this is the candidate?s own work, as part of the standard supervisor declaration.

When including material from publications in a thesis, students should be aware of the copyright policies of journals. It is recommended that students request journals to vary their normal copyright agreements to allow material from an article to be included in a thesis (as the thesis will be publicly available through the University?s Institutional Repository, and from there the National Diet Library database of Japan PhD theses). If, for copyright reasons, material from previously published papers may not be included in the electronically published thesis, the electronically published thesis may cite papers that are already published.

\section{Organization of chapters and sections}

\textbf{Title Page}:  This page is the first page, and should list your name, the thesis title, and the name of your Supervisor and official Co-supervisor, if any, and the year of submission.  Only OIST Faculty or external faculty with whom a Graduate School Co-Supervision agreement exists may be listed as Co-supervisors.

\textbf{Choice of Title}: Select a descriptive and unique title that clearly communicates your research.  Avoid brief or misleading titles.  The title will be displayed on your graduation certificate.  The title should be unique within OIST, to distinguish your thesis from those of others working on similar topics.

\textbf{Declaration of Original Authorship}:  You must declare that the work is your own, and original, by signing and including the declaration.  

\textbf{Co-authorship}:  Co-authors are not permitted on an OIST PhD thesis. All research and analysis should be your own work. Where co-authors have contributed to papers arising from the work, you should not include their data unless it is essential for the scientific narrative.  In such cases, full disclosure of the contribution is required. Acknowledge any work performed by others, whether at OIST or outside OIST.

\textbf{Abstract, Acknowledgements, List of Abbreviations, Glossary, Nomenclature, Dedication, Table of Contents, List of Figures and Tables}: Those are commented directly in the template. Glossary, Nomenclature, and Dedication pages are optional.

\textbf{Main body}:  The main body may be arranged as a single body of material, divided into sub-sections of Introduction (including a statement of the problem), Methods, Results, Discussion, or if preferred, in chapters that each deal with a smaller part of the research, each one itself divided into sub-chapters of Introduction, Methods, Results, Discussion (or similar), as appropriate. 

\textbf{Reference List}:  Provide a complete list of all articles and books cited in your thesis, once only, at the end of the thesis using BibTeX or BibLaTeX. The citations should provide the title of the article, and list at least the first three authors (et al. format is acceptable).  Do NOT include articles not cited in the thesis. Do include ALL articles cited.

\textbf{Appendices}: The examination versions of the thesis must include, as an appendix, published papers; unpublished manuscripts that have been submitted for publication; and manuscripts ready for, or very close to, journal submission.  These should be placed immediately after the final pages of the thesis, and separated from the thesis itself by a single dividing page with the text: ``Previously published articles associated with the research described in this thesis'', or similar. These published papers are included solely for reference by your examiners, and to show that you satisfy the graduation requirement for at least one submitted article.  They will not be included in the on-line version of the thesis once your revisions have been accepted. Papers co-authored during the period of the thesis that do not include material presented in this work should not be included.

\textbf{Appendices and Supplementary Data}: Unlike a journal article, no data or discussion may be presented separately as unpublished supplementary documents or data.  Appendices should be used instead for material that is tangentially relevant to the thesis but does not fit in the main narrative. If you need to refer to large volumes of data that cannot be printed (such as an annotated genome, or a simulation with moving images), lodge the data on an OIST repository or a public database and provide the URL of the dataset in the thesis.  (See also: Reproducibility, above.)


\section{Formatting Requirements}

\textbf{Page size, Margins, Spacing, Justification, Pagination, Header, Fonts}: those are already built-in the template. Do not modify them.

\textbf{Equations}: Equations are considered to be part of the text; they should be formatted consistently throughout the thesis, following the advice of the student's supervisor.  

\textbf{Spelling}: American spelling should be used.

\textbf{Printing}:  Theses submitted for examination should be printed double-spaced on one side of a page only (so that when bound in temporary bindings, the right hand page is the printed page).  Temporary bindings may use any reasonable white bond paper, in A4 size.  Laser printing (in black ink wherever possible, and colour for images where necessary), should be used exclusively, rather than alternatives such as ink-jet, dye sublimation, or wax transfer (for durability of the print).  Final bound copies should use acid-free paper (also known as archival paper) to ensure longevity of the thesis in the collection, and should be printed on both sides of the page (single-spaced, with adjusted margins).

\textbf{Colors}:  Colors may be used in images and charts where necessary to enhance comprehension, but should not be used for normal text or headings.  Avoid the exclusive combination of red and green for binary images, to assist those who have difficulty discriminating hues.  All text should be in black unless color-coding is necessary for meaning or contrast.

\textbf{Figues, Tables, Images}: Those are detailed in a later Chapter with examples.

\textbf{Word length}:  No minimum word length is imposed on OIST graduate theses.  However, be concise in your language and succinct in your expression.  The average length of a PhD thesis will vary between fields and between authors, but typical PhD theses are 100-400 pages in length (approximately 20,000-80,000 words of main body text).  

\textbf{Citations}:  All papers that you reference in your work must be referenced in full using a style relevant to your field.  Provide the full title, a complete list of authors, and the article location and year of publication in the same style for all papers. Use one of several styles you have been introduced to in previous writing.  Refer to papers in the text by either a numerical superscript, a bracketed number or by reference to (Author et al., 1999).  Be consistent in your citation style throughout all sections of your thesis.  Provide a complete list of all papers, books, and proceedings cited in your thesis at the end of the main body of text.  Do not include papers in this list that were not cited in the thesis.  Reference manager software such as Endnote or similar programs that offer ?Cite-While You Write? functionality can assist this process greatly. Use BibTeX or BibLaTeX if you are using LaTeX.

Citing one reference can be done like so: \cite{Lee98} and multiple references in one go like so \cite{Fil09, Muc10, Kra27}.

\textbf{Editing}:  The thesis should be entirely your own work. Minimal editing may be provided by your supervisor(s) or peers but only as a review of initial drafts.  Do not seek assistance from OIST internal or paid external editing services, unless directed to do so by the Dean in revision stages.

\section{Intellectual property and copyright}

The student will retain copyright of the published work, in perpetuity.  The student acknowledges that OIST remains the owner of the intellectual property generated by the research presented in the thesis and that publication of the thesis under the author?s copyright does not diminish this claim.  The thesis will be published online in electronic form within one year of graduation.

 \section{Submission of examination and final versions}

The thesis will be submitted first as a single PDF in a style formatted to assist the examination process. This version should include (inserted after the thesis itself) any papers published during the tenure of the thesis that are relevant to the material therein, for the convenience of the examiners.  Large files should be sent using FileSender, smaller ones less than 10 MB can be sent by email to examination@oist.jp.

The submitted examination thesis should be accompanied by a Declaration from the Supervisor (see Appendix 2 for template). The purpose of the Declaration, a printed sheet which should be completed and signed by the thesis supervisor, is to acknowledge that the work was done in the laboratory of the supervisor, that any coauthors of included material have consented to such inclusion, and that there is no unauthorized use of material for which the copyright is held by other parties. The Declaration will be retained by the Graduate School, and will not be part of the thesis sent for examination or published later.

The two external examiners will examine the written thesis and send their reports to Graduate School, to arrive before the oral.  The oral exam (1 hour presentation, 2 hours closed exam) is then conducted at OIST, and a report prepared by the Chair.  Any required revisions are then sent to the student, and the student then makes the appropriate revisions to the approval of the Supervisor (and maybe checking by examiners as well). Once these are accepted by the Chair, the student is eligible to graduate.

The student should then prepare and submit the final version of the thesis (the final approved PDF without any appended articles or manuscripts, or highlighted areas of revision).  Once this is received and checked by the Graduate School, the degree may be awarded by acclaim of the Faculty Assembly.

The final version must be accompanied by a signed Deposit Consent Form, which confirms to us that you agree to the immediate publication of the thesis online, and provides a checklist for you to complete essentially vouching that the document is free from any copyright restrictions or private information concerns.  You may elect to withhold publication of the main part of the thesis for a set period, with support from the Supervisor, perhaps due to a patent application, or other time-sensitive concern.  The abstract and metadata will be published instead.  Once the reasons for withholding complete publications are no longer relevant, the thesis must be published in full on OIST IR, from whence it is archived in the National Diet Library and accessible online.  No changes can be made once you submit the final version, although the title and other details can be changed prior to that.  The title of the final version is reproduced on the degree certificate. 

