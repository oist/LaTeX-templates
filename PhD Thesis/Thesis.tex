%%%%%%%%%%%%%%%%%%%%%%%%%%%%%%%%%%%%%%%%%
% OIST Doctoral Thesis
% LaTeX Template
% Version 0.3 (2018/03)
%
% Original author:
% Jeremie Gillet
%
%%%%%%%%%%%%%%%%%%%%%%%%%%%%%%%%%%%%%%%%%

%-------------------------------------------------------------------------------
%	REQUIRED PACKAGES AND  CONFIGURATIONS
%-------------------------------------------------------------------------------

\documentclass[temporary]{oist_thesis} % Temporary version for thesis revision and examination
%\documentclass[final]{oist_thesis} % Final version for thesis submission

% The documentclass oist_thesis includes the following packages: geometry, caption, xkeyval

\usepackage[english]{babel} % The document is in English
\usepackage[utf8]{inputenc} % UTF8 encoding
\usepackage[T1]{fontenc} % Font encoding

\usepackage{graphicx} % For including images
\graphicspath{{./Images/}} % Specifies the directory where pictures are stored

\usepackage{eso-pic} % For the background picture on the title page

\usepackage{setspace} % For using single or double spacing
\usepackage{longtable} % tables that can span several pages
\usepackage{pdfpages} % To include a pdf files of your published papers as an appendix
\usepackage{fancyhdr} % For the headers

% For automatically making lists of glossaries and acronyms.
% Make sure that you compile Thesis.tex correctly. If you are compiling with pdflatex,
% you need to run "pdflatex Thesis.tex", "makeglossaries Thesis", and "pdflatex Thesis.tex".
% If you are using overleaf, "you don't have to do anything special, but if you add new terms
% to the glossary once you compiled it, make sure to click on Clear cached files first under
% logs option)" as described in https://overleaf.com/learn/latex/Glossaries.
% If you want glossaries to be clickable, place the following line after \usepackage{hyperref}.
\usepackage[xindy, toc, acronym, symbols, nonumberlist]{glossaries}

\usepackage{hyperref} % Adds clickable links at references

%----------------------------------------------------------------------------------------
%	ADD ABBREVIATIONS, GLOSSARIES, AND ACRONYMS
%----------------------------------------------------------------------------------------

\makeglossaries % Starts glossaries package
%----------------------------------------------------------------------------------------
%	ABBREVIATIONS
%----------------------------------------------------------------------------------------
% For automatic construction of a list of abbreviations, glossaries package is used.
%
% To define an abbreviation (or an acronym), use the following command.
% 		\newacronym[plural=plural_form_of_this_entry]{label}{short}{long}
% Here, label is a lebel of an entry.
% short is used as an abbreviation of long.
% plural=plural_of_this_entry passes a plural form of this entry in texts to the package. plural_of_this_entry is used in texts.
%
% To use this entry in texts, use \gls{label}. If you need a plural form, use \glspl{label}.
% the first letter.
%
% For more details, see the examples below or the following links:
% 	(longest)	http://www.math.ucsd.edu/~jeggers/latex/glossaries-user.pdf
% 	(longer)	https://en.wikibooks.org/wiki/LaTeX/Glossary
% 	(long)		https://ja.overleaf.com/learn/latex/Glossaries

\newacronym[plural=PPTs]{ppt}{PPT}{Positive Partial Transpose}
\newacronym[plural=SRPTs]{srpt}{SRPT}{Schr\"odinger-Robertson Partial Transpose} % Define abbreviations in Preamble/abbreviations.tex. If unnecessary, remove this line.
%----------------------------------------------------------------------------------------
%	GLOSSARY
%----------------------------------------------------------------------------------------
% For automatic construction of a list of glossaries, glossaries package is used.
%
% To define a glossary, use the following command.
% 		\newglossaryentry{label}{
%			name=name_of_this_entry,
%			text=text_of_this_entry,
%			plural=plural_of_this_entry,
%			description=description_of_this_entry
%		}
% Here, label is a lebel of an entry.
% name=name_of_this_entry passes a name of this entry to the package.
% text=text_of_this_entry passes a name of this entry in texts to the package. text_of_this_entry is used in texts.
% plural=plural_of_this_entry passes a plural form of this entry in texts to the package. plural_of_this_entry is used in texts.
% description=description_of_this_entry passes description of this entry to the package.
%
% To use this entry in texts, use \gls{label}. If the first letter needs to be capitalized, use \Gls instead.
% If you need a plural form, use either \glspl{label} or \Glspl{label} depending on whether you want to capitalize
% the first letter.
%
% For more details, see the examples below or the following links:
% 	(longest)	http://www.math.ucsd.edu/~jeggers/latex/glossaries-user.pdf
% 	(longer)	https://en.wikibooks.org/wiki/LaTeX/Glossary
% 	(long)		https://ja.overleaf.com/learn/latex/Glossaries

\newglossaryentry{dipole_blockade}{
	name=Dipole Blockade,
	text=dipole blockade,
	plural=dipole blockades,
	description={Phenomenon in which the simultaneous excitation of two atoms is inhibited by their dipolar interaction}
}

\newglossaryentry{cavity_induced_transparency}{
	name=Cavity Induced Transparency,
	text=cavity induced transparency,
	plural=cavity induced transparencies,
	description={Phenomenon in which a cavity containing two atoms excited with light at a frequency halfway between the atomic frequencies contains the number of photons an empty cavity would contain}
} % Define glossaries in Preamble/glossary.tex. If unnecessary, remove this line.
%----------------------------------------------------------------------------------------
%	NOMENCLATURE
%----------------------------------------------------------------------------------------
% For automatic construction of a list of nomenclatures, glossaries package is used.
%
% To define a nomenclature (or a symbol), use the following command.
% 		\newglossaryentry{label}{
%			type=symbols,
%			name=name_of_this_entry,
%			description=description_of_this_entry,
%			sort=key_for_sort
%		}
% Here, label is a lebel of an entry.
% type=symbols tells glossaries package to which type this entry belongs. Do not modify it.
% name=name_of_this_entry passes a name of this entry to the package.
% description=description_of_this_entry passes description of this entry to the package.
% sort=key_for_sort passes a key of this entry for soting to the package. Strongly encouraged to use.
%
% To use this entry in texts, use \gls{label}. If the first letter needs to be capitalized, use \Gls instead.
%
% For more details, see the examples below or the following links:
% 	(longest)	http://www.math.ucsd.edu/~jeggers/latex/glossaries-user.pdf
% 	(longer)	https://en.wikibooks.org/wiki/LaTeX/Glossary
% 	(long)		https://ja.overleaf.com/learn/latex/Glossaries

\newglossaryentry{light_speed}{
	type=symbols,
	name={\ensuremath{c}},
	description={Speed of light ($2.997\ 924\ 58 \e{8}\ \mbox{ms}^{-1}$)},
	sort=c
}

\newglossaryentry{planck}{
	type=symbols,
	name={\ensuremath{\hbar}},
	description={Planck constant ($1.054\ 572\ 66\e{-34}\ \mbox{Js}$)},
	sort=h
}

\newglossaryentry{boltzmann}{
	type=symbols,
	name={\ensuremath{\hbar}},
	description={Boltzmann constant ($1.380\ 658\e{-23}\ \mbox{JK}^{-1} $)},
	sort=kb
}

\newglossaryentry{free_spece_impedance}{
	type=symbols,
	name={\ensuremath{Z_0}},
	description={Impedance of free space  ($376.730\ 313\ 461\ \Omega) $},
	sort=zzero
}

\newglossaryentry{free_space_permeability}{
	type=symbols,
	name={\ensuremath{\mu_0}},
	description={Permeability of free-space ($4\pi\e{-7}\ \mbox{Hm}^{-1}$)},
	sort=muzero
} % Define nomenclature in Preable/nomenclature.tex. If unnecessary, remove this line.

%----------------------------------------------------------------------------------------
%	ADD YOUR PACKAGES (be careful of package interaction)
%----------------------------------------------------------------------------------------

\usepackage{amsthm,amsmath,amssymb,amsfonts,bbm}% Math symbols

%----------------------------------------------------------------------------------------
%	ADD YOUR DEFINITIONS AND COMMANDS
%----------------------------------------------------------------------------------------

% Example of New Commands
\newcommand{\bea}{\begin{eqnarray}} % Shortcut for equation arrays
\newcommand{\eea}{\end{eqnarray}}
\newcommand{\e}[1]{\times 10^{#1}}  % Powers of 10 notation

% Example of Defining a theorem box for Criteria 
\newtheorem{critere}{Criterion}
\newcommand{\crit}[2]{
	\begin{center}
		\fbox{ \begin{minipage}[c]{0.9 \textwidth}
				\begin{critere}
					\textbf{\textup{ #1}} --- #2
				\end{critere}
	\end{minipage}  } \end{center}
}

%----------------------------------------------------------------------------------------
%	PICK YOUR BIBLIOGRAPHY STYLE
%----------------------------------------------------------------------------------------

\usepackage[square, numbers, sort&compress]{natbib} % for bibliography - Square brackets, citing references with numbers, citations sorted by appearance in the text and compressed (as in [4-7])
%\usepackage[longnamesfirst,round]{natbib} % Natural Sciences bibliography

%\bibliographystyle{Preamble/physics_bibstyle} % You may use a different style adapted to your field
\bibliographystyle{abbrvnat} % You may use a different style adapted to your field

%-------------------------------------------------------------------------------
%	TITLE PAGE
%-------------------------------------------------------------------------------

\begin{document}
\pagestyle{empty} % No page numbers
\frontmatter % Use roman page numbering style (i, ii, iii, iv...) for the preamble pages

\puttitle{
	title=\LaTeX\ Thesis Template, % Title of the thesis
	name=Jeremie Gillet, % Author name
	supervisor=S.~Upervisor, % Supervisor name
	cosupervisor=C.~O'Supervisor, % Co-Supervisor name, remove this line if there is none
	submissiondate={March 2018}  % Submission date "Month, year"
}

%-------------------------------------------------------------------------------
%	PREAMBLE PAGES (delete unnecessary pages)
%-------------------------------------------------------------------------------

\startpreamble

\input{Preamble/declaration}
\input{Preamble/abstract}
\input{Preamble/acknowledgments}
\printacronyms[title=Abbreviation, toctitle=Abbreviation]
\printglossary[title=Glossary, toctitle=Glossary]
\printsymbols[title=Nomenclature, toctitle=Nomenclature]
\input{Preamble/dedication}

%-------------------------------------------------------------------------------
%	LIST OF CONTENTS/FIGURES/TABLES
%-------------------------------------------------------------------------------

\unnumberedchapter{Contents}
\tableofcontents % Write out the Table of Contents
\unnumberedchapter{List of Figures}
\listoffigures % Write out the List of Figures
\unnumberedchapter{List of Tables}
\listoftables % Write out the List of Tables

%-------------------------------------------------------------------------------
%	THESIS MAIN TEXT
%-------------------------------------------------------------------------------

\addtocontents{toc}{\vspace{2em}} % Add a gap in the Contents, for aesthetics
\mainmatter % Begin numeric (1,2,3...) page numbering

\unnumberedchapter{Introduction} % Title of the unnumbered chapter
\input{MainText/introduction} % Introduction (unnumbered)

\numberedchapter % Regular chapters following
\input{MainText/chapter1} % Input your chapters here
\chapter{How to use the templates} \label{ch-2}

This is a practical guide into how to use this template, by explaining the role of the different folders, how to use glossaries package, and an option of \verb|\documentclass{oist_thesis}|, which accepts either \verb|temporary| or \verb|final|.

\section{Folders}

The main folder contains three folders detailed here:

\begin{itemize}

\item \textbf{Images.} This folder should contain all the images that you will use in your thesis. It can contain subfolders, for example one for each chapter. To include an image from the main text, use something like \texttt{\textbackslash includegraphics\{subfolder/image.jpg\} } without worrying about the \texttt{Images} path.

\item \textbf{MainText.} This folder contains a series of \LaTeX\ files that form the main text: introduction, chapters, conclusion, appendices and published articles. The introduction and conclusion as they are now are not numbered, which creates a few difficulties with the headers of the thesis. Those are solved by including the commands \texttt{\textbackslash unnumberedchapter\{\}} and \texttt{\textbackslash numberedchapter} before including the files in \texttt{xxx\_Thesis.tex}. If you want the introduction and conclusion to be numbered, re-write and treat them as regular chapters.

\item \textbf{Preamble.} This folder contains a series of \LaTeX\ files with the pages that will appear before the main text. Please write (or copy and paste) your own text in those files and delete the dummy text when appropriate. The files are:
\begin{itemize}
\item \texttt{abbreviations.tex} --- List of abbreviations.
\item \texttt{abstract.tex} --- Abstract. Follow directions in the file.
\item \texttt{acknowledgments.tex} --- Acknowledgments. Follow directions in the file.
\item \texttt{declaration.tex} --- Declaration of Original and Sole Authorship. Only modify the last item. This page needs to be signed once printed.
\item \texttt{dedication.tex} --- Dedication (optional). Should only be a very few lines.
\item \texttt{glossary.tex} --- Glossary (optional).
\item \texttt{nomenclature.tex} --- Nomenclature (optional).
\item \texttt{physics\_bibstyle.bst} --- Bibliography style file modified by Jeremie Gillet in 2011 to suit his thesis. Might be suitable for physics. If you want to use another custom bibliography style, include the file in this folder.
\item \texttt{Thesis\_bibliography.bib} --- BibTeX file containing your bibliography.
\end{itemize}

\end{itemize}

\section{The Usage of Glossaries Package}

Glossaries package automatically creates lists of glossaries, nomenclatures, and abbreviations with indices. Furthermore, it can automatically find first use of a term, and appropriately handle it. For example, \verb|\gls{ppt}| prints out \gls{ppt} when used for the first time. However, it prints out \gls{ppt} for the second time. The package can also handle nomenclatures. For example, \verb|\gls{light_speed}| prints out \gls{light_speed}.

For details on how to define glossaries, nomenclatures, and abbreviations, refer to \texttt{glossary.tex}, \texttt{nomenclature.tex}, and \texttt{abbreviations.tex} in \texttt{Preamble} folder, respectively. You can find descriptions as well as examples.

To use a glossary, use \verb|\gls{label}|, where \texttt{label} is an identifier of this glossary defined in \texttt{Preamble/glossary.tex}. For example, \verb|\gls{dipole_blockade}| prints out \gls{dipole_blockade}. If you need to capitalize the first letter, use \verb|\Gls{dipole_blockade}| instead. It prints out \Gls{dipole_blockade}. If you need a plural form, use \verb|\glspl{dipole_blockade}|. If you need a plural form with its first letter capitalized, use \verb|\Glspl{dipole_blockade}|.

When compiling a tex file, this glossaries package requires a special care. If you compile \texttt{Thesis.tex} with pdflatex on your local machine, you need to run the following commands:
\begin{enumerate}
	\item \texttt{pdflatex Thesis.tex}
	\item \texttt{makeglossaries Thesis}
	\item \texttt{pdflatex Thesis.tex}
\end{enumerate}
If you are using Overleaf, "you don't have to do anything special, but if you add new terms to the glossary once you compiled it, make sure to click on Clear cached files first under logs option)" as described in \url{https://overleaf.com/learn/latex/Glossaries}.

\section{\texttt{Thesis.tex}}

This is the main file, the only one that needs to be compiled to build the thesis. Compile once with \LaTeX, once with BibTeX and finally twice with \LaTeX\ to get all the references right. At the top of this file, you can see \verb|\documentclass[temporary]{oist_thesis}|. When you submit a temporary version to the graduate school, do not modify it. When you submit a final version, use \verb|\documentclass[final]{oist_thesis}| instead. 

Let's go through each section and comment them briefly. The last section will emphasize the differences between options \verb|\documentclass[temporary]{oist_thesis}| and \verb|\documentclass[final]{oist_thesis}|.

\subsection{PACKAGES AND OTHER DOCUMENT CONFIGURATIONS}

This section contains the minimum number of packages and definitions to compile the thesis. No line should be removed or modified.

\subsection{ADD YOUR CUSTOM VALUES, COMMANDS AND PACKAGES}

This section should not be modified directly. Instead, your packages and definitions should be included in  \texttt{Preamble/mydefinitions.tex}.

\subsection{TITLE PAGE}

Creates the title page. Do not modify.

\subsection{PREAMBLE PAGES}

Structures the style (header) for the preamble pages and builds them. Do not modify, except for deleting the optional preambles you might not want to include.

\subsection{LIST OF CONTENTS/FIGURES/TABLES}

Creates the different lists. Do not modify.

\subsection{THESIS MAIN TEXT}

Structures the style for the main text chapters and builds them. 

The command \texttt{\textbackslash numberedchapter} is only relevant for a transition between unnumbered sections and numbered sections, it does not need to be included between each chapter. 

\subsection{APPENDICES}

Structures the style for the appendices and builds them. The appendices are numbered with letters but are structured like regular chapters.

\subsection{BIBLIOGRAPHY}

Builds the bibliography. The style of the bibliography can be defined in \texttt{Preamble/mydefinitions.tex}.

\subsection{PUBLISHED ARTICLES}

This last section add the PDF files of your previously published articles (or about to be published) to the thesis. You should only include PDF files provided by the publishing journal. This is strictly for the examiners' convenience in the temporary bound thesis, as for copyright reasons these files may not be published in the final version of the thesis.

\subsection{Differences between a temporary version and final version}

There are two main differences between \verb|\documentclass[temporary]{oist_thesis}| and \verb|\documentclass[final]{oist_thesis}|. 

The first difference is that the final version (\verb|\documentclass[final]{oist_thesis}|) does not contain the published articles for copyright reasons. 

The second difference is in the document style: page size, header and line spacing are different This might create small issues, such as page breaking with large tables, images or captions, when compiling the same content.




\input{MainText/chapter3}
%\input{MainText/chapter4}
%\input{MainText/chapter5}

\unnumberedchapter{Conclusion} % Title of the unnumbered chapter
\input{MainText/conclusion} % Conclusion (unnumbered)

%-------------------------------------------------------------------------------
%	APPENDICES
%-------------------------------------------------------------------------------
\addtocontents{toc}{\vspace{2em}} % Add a gap in the Contents, for aesthetics
\appendix

\numberedchapter % Regular chapters following
\input{MainText/appendixA}
\input{MainText/appendixB}
%\input{MainText/appendixC}

%-------------------------------------------------------------------------------
%	BIBLIOGRAPHY
%-------------------------------------------------------------------------------

\addtocontents{toc}{\vspace{2em}} % Add a gap in the Contents, for aesthetics
\unnumberedchapter{Bibliography} % Title of the unnumbered chapter
\bibliography{Preamble/Thesis_bibliography} % The references information are stored in the file named "Thesis_bibliography.bib"

%-------------------------------------------------------------------------------
%	PUBLISHED ARTICLES (only appears in temporary thesis)
%-------------------------------------------------------------------------------

\includepublications

\end{document}
