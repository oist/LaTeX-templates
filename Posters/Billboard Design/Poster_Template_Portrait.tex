% Poster Template for the Okinawa Institute of Science and Technology (OIST)
% Created by Jeremie Gillet on June 2019, with help from Tani Unit members
% Inspired from https://www.youtube.com/watch?v=1RwJbhkCA58

\documentclass[
    a0paper, % Size of poster
    portrait, % Orientation
    fontscale=0.4 % General scaling for fonts, increase the number for smaller fonts (considered removing text first)
    ]{baposter}

% Packages required for the template
\usepackage{qrcode} % For creating QR codes
\usepackage{fix-cm} % Arbitrary font size

% Add any packages you need here
\usepackage{graphicx} % Required for including images
\graphicspath{{figures/}} % Directory in which figures are stored
\usepackage{amsmath} % For typesetting math

% Defining colors
\selectcolormodel{HTML}
\definecolor{oistRed}{HTML}{C80019} 
\definecolor{grey}{HTML}{0F0D08} 

% Changing fonts family
\renewcommand{\familydefault}{\sfdefault}
\usepackage{avant}

% Starting document, feel free to change the style of headers and boxes
% Documentation can be found here: http://www.brian-amberg.de/uni/poster/
% Unfortunately it is not very complete
\begin{document}
\begin{poster}{ % General poster options
    columns=3, % Number of columns
    eyecatcher=true, % Set to false for ignoring the left logo in the title and move the title left
    colspacing=1em, % Column spacing
    bgColorOne=white, % Background color for the gradient on the left side of the poster
    bgColorTwo=white, % Background color for the gradient on the right side of the poster
    borderColor=oistRed, % Border color
    headerheight=0.01 \textheight, % Height of the header
    headershape=roundedright, % Specify the rounded corner in the content box headers, can be: rectangle, small-rounded, roundedright, roundedleft or rounded
    headerfont=\Large\bf\textsc, % Large, bold and sans serif font in the headers of content boxes
    headerborder=closed, % Adds a border around the header of content boxes
    linewidth=2pt, % Width of the border lines around content boxes
    headerColorOne=white, % Background color for the header in the content boxes (left side)
    headerColorTwo=oistRed, % Background color for the header in the content boxes (right side)
    headerFontColor=grey, % Text color for the header text in the content boxes
    boxColorOne=white, % Background color of the content boxes
    textborder=roundedleft % Format of the border around content boxes, can be: none, bars, coils, triangles, rectangle, rounded, roundedsmall, roundedright or faded
}{}{}{}{} % We don't use the poster header features


% Center message
\begin{posterbox}[
    name = message,  % Name for alignment 
    column = 0, % Second column 
    span = 3, % Over 3 columns
    boxColorOne=oistRed, % Changing background color
    headershade=plain,  % Changing header style
    headerColorOne=oistRed % Changing header color
    ]{}
    \fontsize{45}{50} \selectfont  \color{white} % Changing font size and color
    
    \includegraphics[width=\textwidth, trim= 3 0 3 10]{footer-white.png}

    \hspace{1em}
    \begin{minipage}[c][ 0.4 \textheight ]{0.6\textwidth}

        Single sentence explaining your  \textbf{main message} in plain English, \textbf{no jargon} allowed

        \vspace{2em}
        
        \LARGE 
        List of authors\\ 
        Okinawa Institute of Science and Technology, Japan
    \end{minipage}

    % Generating the QR code. Alternatively, they can be generated online with tools such as https://www.qrcode-monkey.com
    \vspace{-5em} \hfill 
    \colorbox{white}{ \makebox[4.5em]{
        \color{black} 
        \qrcode[height = 4em,padding]{https://groups.oist.jp/grad} }
    } \hspace{0.3em}

    \vspace{0.3em}
    \hfill {\normalsize Reference to paper, URL, contact email, content of QR code... }
    \vspace{0.3em}

\end{posterbox}



% Content for left column
\begin{posterbox}[
    name = box1,  % Name for alignment 
    column = 0, % First column 
    below = message % Alignment
    ]{Box title}
    Could be introduction, abstract, methods, results, discussion...
\end{posterbox}


\begin{posterbox}[
    name = box2,  % Name for alignment 
    column = 1, % Second column 
    below = message % Alignment
    ]{Box title}
    Could be introduction, abstract, methods, results, discussion...
\end{posterbox}


\begin{posterbox}[
    name = box3,  % Name for alignment 
    column = 2, % Third column 
    below = message % Alignment
    ]{Box title}
    Could be introduction, abstract, methods, results, discussion...
\end{posterbox}


\begin{posterbox}[
    name = box4,  % Name for alignment 
    column = 0, % First column 
    below = box1 % Alignment
    ]{Box title}
    Could be introduction, abstract, methods, results, discussion...
\end{posterbox}


% Content for right column
\begin{posterbox}[
    name = box5,  % Name for alignment 
    column = 1, % Last column 
    below = box2 % Alignment
    ]{Box title}
    Could be introduction, abstract, methods, results, discussion...
\end{posterbox}

% References, if required
\begin{posterbox}[
    name = references,  % Name for alignment 
    column = 2, % Last column 
    below = box3 % Alignment
    ]{References}
\renewcommand{\section}[2]{} % Get rid of the default "References" section title
\nocite{*} % Insert publications even if they are not cited in the poster
\bibliographystyle{unsrt}
\bibliography{sample} % Use sample.bib as the bibliography file
\end{posterbox}

\end{poster}
\end{document}